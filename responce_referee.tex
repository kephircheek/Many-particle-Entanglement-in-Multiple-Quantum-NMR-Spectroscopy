We thank the Referee for useful comments. We expanded the Introduction and Conclusion to discuss our motivations in more details. We also included a formal definition of many-particle entanglement in the revised manuscript. We discuss these two changes in detail below. 
\begin{enumerate}
\item To discuss motivation in more detail we added two paragraphs (the second and fourth paragraphs) to the introduction and added text to the sixth and seventh paragraphs of the introduction. We also added text in the last section of the manuscript.

The new second paragraph of the Introduction is:

Attempts to quantify entanglement are motivated by the
desire to understand and quantify resources responsible for
advantages  of quantum computing over classical computing. Pair entanglement is the most familiar while many-particle entanglement is its most general extension.
The existing applications of  MQ NMR to quantum information make it important to understand many-particle entanglement in the MQ NMR context. The starting point of such investigation must be the simplest model with non-trivial behavior. This is the motivation of the present work.

The new fourth paragraph  is:

In order to investigate many-spin entanglement it is necessary to work out a model of interacting spins, in which many-spin dynamics can be studied at low temperatures. 
It is also important that the model contains a sufficiently large number of spins and is applicable at arbitrary temperatures.
Only then it is possible to investigate many-spin entanglement and its dependence on the temperature.

The additional  text in the sixth paragraph of the Introduction is:

We discover that the system exhibits $k$-spins entanglement with $k$ growing as temperature decreases. 
Almost all spins are entangled at low temperatures despite the absence of  entanglement in the initial state. We expect this behavior to be generic for MQ NMR. 

The new text in the seventh paragraph of the Introduction is:

We perform all calculations for a system of 201 spins. 
Such an investigation of many-spin entanglement is performed for the first time.
In principle, analogous calculations can be performed for  system with several thousand  spins. 

The following text was added in the Conclusion section:

The main lesson consists in significant growth of many-particle entanglement at low temperatures. 
All or almost all spins are entangled at the dimensionless temperature 1/b of the order of 1. 
This suggests that k-entangled states with large k emerge in a typical NMR systems at low temperatures. 
This is particularly interesting given the absence of entanglement in the initial state.

\item We give a definition of many-particle entanglement in Section IV on page 5:
We give now a definition of many-particle entanglement [30]. 
A pure state is $k$-particle entangled, if it can be written as a product \mbox{$\left| \Psi_\mathrm{k-ent} \right\rangle = \otimes^M_{l=1} \left| \Psi_l \right\rangle$}, where $\left| \Psi_l \right\rangle$ is a state of $N_l$ particles \mbox{($\sum\limits_{l=1}^M N_l = N$)}, each  $\left| \Psi_l \right\rangle$ does not factorize, and the maximal $N_l \geq k$. A generalisation for mixed states is straightforward[30].

[30] P. Hyllus, W. Laskowski, R. Krischek, C. Schwemmer,
W. Wieczorek, H. Weinfurter, L. Pezz ́e, and A. Smerzi, Phys. Rev. A 85, 022321 (2012).
\end{enumerate}



	