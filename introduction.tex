\section{Introduction}
Multiple quantum (MQ) NMR spectroscopy \cite{mq_nmr_experiment} was introduced for the investigation of nuclear spin distributions in various materials (liquid crystals \cite{spin_distribution_in_liquid_system}, simple organic systems \cite{mq_nmr_experiment}, amorphous hydrogenated silicon \cite{spin_distribution_in_silicon}, etc.). 
It also turned out to be useful for probing the decoherence rate in highly correlated spin clusters \cite{decoherence_register,decoherence_ca_f2}. 
The scaling of the decoherence rate with the number of the correlated spins has also been demonstrated \cite{decoherence_register,lab:decoherence_2018}. 
Essentially, the MQ NMR dynamics is a suitable method to quantify the development of MQ coherences starting from the $z$-polarisation and ending with a collective state of all spins. 
The method allows us to describe the spreading of correlations \cite{mq_nmr_experiment,spin_distribution_in_liquid_system,decoherence_under_dq,nmr_dyn} and offers a signature of localisation effects \cite{loc_deloc_nmr_dyn,loc_in_chain}. The spreading rate can be described through out-of-time ordered correlations (OTOCs) which are connected with the distribution of MQ NMR coherences. 
\par
Attempts to quantify entanglement are motivated by the
desire to understand and quantify resources responsible for
advantages  of quantum computing over classical computing.
Pair entanglement is the most familiar while many-particle entanglement is its most general extension.
The existing applications of  MQ NMR to quantum information make it important to understand many-particle entanglement in the MQ NMR context.  
The starting point of such investigation must be the simplest model with non-trivial behavior. This is the motivation of the present work.
\par
Connections between MQ coherences and entanglement have been established only for spin pairs \cite{lab:entanglement_dyn_2003,entanglement_dyn,nuclear_polarization_and_entanglement}.  The same is true for the MQ coherences as a witness of entanglement \cite{sep_of_mixed_states}. The MQ NMR coherence of the second order was used for the construction of an entanglement witness for a two-spin system with the dipole-dipole interactions (DDIs) \cite{lab:entanglement_witness_nmr_2008,lab:mq_mnr_qinfo_2012}. At the same time, MQ NMR dynamics allows us to clarify deeper connections between MQ coherences and entanglement. Those connections are closely related to the spread of MQ correlations inside a many-spin system in the evolution process. As a result, it is possible to extract information about many-qubit entanglement and entanglement witnesses from the second moment of the intensity spectrum of the MQ NMR coherences \cite{otoc_to_enanglement_via_mqcoh}. It is also important that there is a relationship between the second moment of MQ NMR coherences and the quantum Fisher information \cite{qmetrology_for_qinfo,qmetrology_nonclassiscal_state}. In particular, it was shown that the second moment of the MQ NMR spectrum provides a lower bound on the quantum Fisher information \cite{otoc_to_enanglement_via_mqcoh}.
\par
In order to investigate many-spin entanglement it is necessary to work out a model of interacting spins, in which many-spin dynamics can be studied at low temperatures. 
It is also important that the model contains a sufficiently large number of spins and is applicable at arbitrary temperatures.
Only then it is possible to investigate many-spin entanglement and its dependence on the temperature.
\par
One would think that MQ NMR in one-dimensional systems is most suitable for the investigation of many-spin entanglement because a consistent quantum-mechanical theory of MQ NMR dynamics has been developed only for one-dimensional systems \cite{lab:nmr_dyn_1996,lab:mq_nmr_in_chain_1997,lab:mq_dyn_of_chain_in_solid_2000}. However, this is not the case. The point is that the exact solutions for MQ NMR dynamics of one-dimensional systems demonstrate \cite{lab:nmr_dyn_1996,lab:mq_nmr_in_chain_1997,lab:mq_dyn_of_chain_in_solid_2000} that, starting from a thermodynamic equilibrium state, only zero and double quantum coherences are produced in the approximation of the nearest neighbour interactions. As a result, the second moment (dispersion) of the MQ NMR spectrum is small and many-qubit entanglement does not appear.
\par
For the investigation of many-qubit entanglement, we build on the model \cite{nanopore_model} of a non-spherical nanopore filled with a gas of spin-carrying atoms (for example, xenon) or molecules in a strong external magnetic field. 
It is well known that the dipole-dipole interactions (DDIs) of spin-carrying atoms (molecules) in such nanopores do not average out to zero due to molecular diffusion \cite{nanopore_model,lab:depolar_in_nanocavities_2004}. 
It is very significant that the residual averaged DDIs are determined by only one coupling constant, which is the same for all pairs of interacting spins \cite{nanopore_model,lab:depolar_in_nanocavities_2004}. 
This means that essentially we have a system of equivalent spins and its MQ NMR dynamics can be investigated exactly \cite{lab:mq_nmr_dyn_in_nanopores_2009}.
For this model OTOCs allow the investigation of many-particle entanglement, and the extraction of information about the number of the entangled spins during the system evolution. 
The temperature dependence of the number of the entangled spins can be also investigated.
We discover that the system exhibits $k$-spin entanglement with $k$ growing as the temperature decreases. 
Almost all spins are entangled at low temperatures despite the absence of  entanglement in the initial state. We expect this behavior to be generic for MQ NMR.
\par
The existing theoretical approach \cite{lab:mq_nmr_dyn_in_nanopores_2009} to the MQ NMR dynamics of a system of equivalent spins is valid only in the high-temperature region. In order to investigate the many-qubit entanglement, we develop a theory of the MQ NMR dynamics of equivalent spins at low temperatures. 
We perform all calculations for a system of 201 spins. 
Such an investigation of many-spin entanglement is performed for the first time.
In principle, analogous calculations can be performed for  systems with several thousand  spins. 
\par
The present paper investigates the connection of the second moment of the MQ NMR spectrum of spin-carrying atoms (molecules) in a nanopore with many-spin entanglement in the system. The paper is organised as follows. In Sec. \ref{sec:mq_dyn} the theory of MQ NMR dynamics at low temperatures in the system of equivalent spins coupled by the DDIs is developed. An analytical solution for the MQ NMR dynamics of a three-spin system is obtained in Sec. \ref{sec:exact_sol}. The second moment (dispersion) of the MQ NMR spectrum of the system of equivalent spins in a nanopore at arbitrary temperatures is obtained in Sec. \ref{sec:second_moment}. The investigation of the dependence of  many-spin entanglement on the temperature is given in Sec. \ref{sec:entanglement}. We briefly summarise our results in the concluding Sec. \ref{sec:conslusions}.
