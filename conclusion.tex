\section{Conclusion} 
\label{sec:conslusions}
We investigated many-particle entanglement in MQ NMR spectroscopy using a nanocavity filled with spin-carrying atoms (molecules). 
We developed a theory of MQ NMR in a nanocavity at low temperatures. 
The theory is based on the idea that  molecular diffusion is substantially faster than the time of the spin flip-flop processes. 
As a result, the problem is reduced to a system of equivalent spins [23, 25], which can be analyzed in the basis of the common eigenstates of the total spin angular momentum and its projection on the external magnetic field. 
Since there is a connection between the second moment (dispersion) of the distribution of the MQ NMR intensities and many-spin entanglement [17], we extracted information about many-spin entanglement from the MQ NMR spectrum. The temperature dependence of many-spin entanglement was also investigated.
\par
The main lesson consists in significant growth of many-particle entanglement at low temperatures. 
All or almost all spins are entangled at the dimensionless temperature $\frac{1}{b}$ of the order of 1. 
This suggests that $k$-entangled states with large $k$ emerge in a typical MQ NMR system at low temperatures. 
This is particularly interesting given the absence of entanglement in the initial state. We expect such behavior to be typical for MQ NMR. 
\par
We can conclude that MQ NMR spectroscopy is an effective method for the investigation of many-spin entanglement and the spreading of MQ correlations inside many-spin systems. It can be used for experimental investigations of quantum information processing in solids (note a related study of decoherence in liquids \cite{HOU2017863}).
\par 
